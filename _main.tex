% Options for packages loaded elsewhere
\PassOptionsToPackage{unicode}{hyperref}
\PassOptionsToPackage{hyphens}{url}
%
\documentclass[
]{book}
\usepackage{amsmath,amssymb}
\usepackage{lmodern}
\usepackage{iftex}
\ifPDFTeX
  \usepackage[T1]{fontenc}
  \usepackage[utf8]{inputenc}
  \usepackage{textcomp} % provide euro and other symbols
\else % if luatex or xetex
  \usepackage{unicode-math}
  \defaultfontfeatures{Scale=MatchLowercase}
  \defaultfontfeatures[\rmfamily]{Ligatures=TeX,Scale=1}
\fi
% Use upquote if available, for straight quotes in verbatim environments
\IfFileExists{upquote.sty}{\usepackage{upquote}}{}
\IfFileExists{microtype.sty}{% use microtype if available
  \usepackage[]{microtype}
  \UseMicrotypeSet[protrusion]{basicmath} % disable protrusion for tt fonts
}{}
\makeatletter
\@ifundefined{KOMAClassName}{% if non-KOMA class
  \IfFileExists{parskip.sty}{%
    \usepackage{parskip}
  }{% else
    \setlength{\parindent}{0pt}
    \setlength{\parskip}{6pt plus 2pt minus 1pt}}
}{% if KOMA class
  \KOMAoptions{parskip=half}}
\makeatother
\usepackage{xcolor}
\IfFileExists{xurl.sty}{\usepackage{xurl}}{} % add URL line breaks if available
\IfFileExists{bookmark.sty}{\usepackage{bookmark}}{\usepackage{hyperref}}
\hypersetup{
  pdftitle={Today I Learned in 2022},
  pdfauthor={PARK Yeonkyu},
  hidelinks,
  pdfcreator={LaTeX via pandoc}}
\urlstyle{same} % disable monospaced font for URLs
\usepackage{longtable,booktabs,array}
\usepackage{calc} % for calculating minipage widths
% Correct order of tables after \paragraph or \subparagraph
\usepackage{etoolbox}
\makeatletter
\patchcmd\longtable{\par}{\if@noskipsec\mbox{}\fi\par}{}{}
\makeatother
% Allow footnotes in longtable head/foot
\IfFileExists{footnotehyper.sty}{\usepackage{footnotehyper}}{\usepackage{footnote}}
\makesavenoteenv{longtable}
\usepackage{graphicx}
\makeatletter
\def\maxwidth{\ifdim\Gin@nat@width>\linewidth\linewidth\else\Gin@nat@width\fi}
\def\maxheight{\ifdim\Gin@nat@height>\textheight\textheight\else\Gin@nat@height\fi}
\makeatother
% Scale images if necessary, so that they will not overflow the page
% margins by default, and it is still possible to overwrite the defaults
% using explicit options in \includegraphics[width, height, ...]{}
\setkeys{Gin}{width=\maxwidth,height=\maxheight,keepaspectratio}
% Set default figure placement to htbp
\makeatletter
\def\fps@figure{htbp}
\makeatother
\setlength{\emergencystretch}{3em} % prevent overfull lines
\providecommand{\tightlist}{%
  \setlength{\itemsep}{0pt}\setlength{\parskip}{0pt}}
\setcounter{secnumdepth}{5}
\usepackage{booktabs}
\ifLuaTeX
  \usepackage{selnolig}  % disable illegal ligatures
\fi
\usepackage[]{natbib}
\bibliographystyle{plainnat}

\title{Today I Learned in 2022}
\author{PARK Yeonkyu}
\date{2022-02-20}

\begin{document}
\maketitle

{
\setcounter{tocdepth}{1}
\tableofcontents
}
\hypertarget{uxc2dcuxc791uxd558uxba70}{%
\chapter{시작하며}\label{uxc2dcuxc791uxd558uxba70}}

1/01 github에 매일매일 잔디 심고 있었다.\\
2/18 github를 청소해 보겠다고 repo를 전체 삭제했다.\\
잔디도 깔끔하게 같이 사라졌다.
다시 시작하는걸로\ldots{}

\hypertarget{uxb144-1uxc6d4}{%
\chapter{2022년 1월}\label{uxb144-1uxc6d4}}

\protect\hyperlink{github-repo-uxc0aduxc81cuxd558uxba74-uxc794uxb514uxb3c4-uxc8fduxb294uxb2e4}{2월 18일}, 날짜도 참\ldots{} 심어 두었던 잔디를 모두 날려 먹었다.\\
Good Luck

\hypertarget{uxb144-2uxc6d4}{%
\chapter{2022년 2월}\label{uxb144-2uxc6d4}}

\hypertarget{til20220218}{%
\section{TIL20220218}\label{til20220218}}

\hypertarget{github-repo-uxc0aduxc81cuxd558uxba74-uxc794uxb514uxb3c4-uxc8fduxb294uxb2e4}{%
\subsection{github repo 삭제하면 잔디도 죽는다}\label{github-repo-uxc0aduxc81cuxd558uxba74-uxc794uxb514uxb3c4-uxc8fduxb294uxb2e4}}

사용하던 github repo를 모두 삭제했다.\\
repo가 삭제되면 심어 두었던 잔디도 같이 죽는다는 것을 알게 되었다.\\
덕분에 깨끗한 마음으로 새로 시작할 수 있을 것 같다 `\^{}'

\includegraphics{https://user-images.githubusercontent.com/72383349/154695072-8480a46d-4174-4a4a-829f-e97f361dbf29.png}

\hypertarget{bookdownuxc73cuxb85c-uxbe14uxb85cuxadf8-uxc804uxd658}{%
\subsection{bookdown으로 블로그 전환}\label{bookdownuxc73cuxb85c-uxbe14uxb85cuxadf8-uxc804uxd658}}

jekyll로 잘 운영하던 블로그를 bookdown으로 이사, 일단 가볍고 문서 작성하기 편하고 체계적이고, 무엇보다 기존 repo 싹 날려서 그렇다.\\
RStudio, Bookdown, Github pages 등으로 검색하면 여러 정보를 얻을 수 있다. 중간중간 설명이 필요한 부분도 있지만 우선 \href{https://unfinishedgod.netlify.app/2020/08/04/bookdowngithub-page\%EC\%97\%90-publish\%ED\%95\%98\%EA\%B8\%B0/}{여기} 참조하면 세팅할 수 있다.

\hypertarget{til2022019}{%
\section{TIL2022019}\label{til2022019}}

\hypertarget{netlifyuxc640-bookdown-uxc5f0uxb3d9}{%
\subsection{netlify와 bookdown 연동}\label{netlifyuxc640-bookdown-uxc5f0uxb3d9}}

\url{https://bookdown.org/yihui/blogdown/netlify.html}

  \bibliography{book.bib,packages.bib}

\end{document}
